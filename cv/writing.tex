%-------------------------------------------------------------------------------
%	SECTION TITLE
%-------------------------------------------------------------------------------
\cvsection{Conférences et écrits}


%-------------------------------------------------------------------------------
%	CONTENT
%-------------------------------------------------------------------------------
\begin{cventries}

%---------------------------------------------------------
  \cventry
    {Auteur, chapitre de livre} % Role
    {Gestion simultanée de l’authentification et de la confiance par blockchain dans les réseaux décentralisés} % Title
    {} % Location
    {Juillet 2018} % Date(s)
    {
      \begin{cvitems} % Description(s)
        \item {Description et performances du framework BATM (Blockchain Authentication and Trust Module) développé au cours de ma thèse}
      \end{cvitems}
    }

%---------------------------------------------------------
  \cventry
    {Co-auteur, article de conférence WIMOB 2017, ROME} % Role
    {Completely independent spanning trees for enhancing the robustness in ad-hoc Networks} % Title
    {} % Location
    {Octobre 2017} % Date(s)
    {
      \begin{cvitems} % Description(s)
        \item {\'Etude sur la faisabilité du calcul d'arbres disjoints dans les réseaux ad-hoc, dans le but d'augmenter la résistance aux pannes du réseau et d'optimiser la sélection des routes réseaux en fonction de la priorité et du type de services déployé.}
      \end{cvitems}
    }

%---------------------------------------------------------
  \cventry
    {Conférencier, conférence Big Block Theory, ESCP PARIS} % Role
    {Blockchain, authentification et confiance dans l'IoT}
    {} % Location
    {Mars 2017} % Date(s)
    {
      \begin{cvitems} % Description(s)
      \item {Présentation du framework BATM développé durant ma thèse à la conférence Big Block Theory}
      \end{cvitems}
    }

% ---------------------------------------------------------
  \cventry
    {Auteur, article en libre accès} % Role
    {Blockchain based trust \& authentication for decentralized sensor networks} % Title
    {} % Location
    {Février 2017} % Date(s)
    {
      \begin{cvitems} % Description(s)
        \item {\'Etude préliminaire concernant l'intérêt des blockchains pour les réseaux de capteurs et l'IoT. Définition des contraintes et caractéristiques liées aux réseaux décentralisés et première description d'un système de gestion de confiance qui donnera BATM par la suite}
      \end{cvitems}
    }

% ---------------------------------------------------------
  \cventry
    {Co-auteur, article de conférence WASC 2016, DIJON} % Role
    {WiseEye: A Platform to Manage and Experiment on Smart Camera Networks} % Title
    {} % Location
    {Juin 2016} % Date(s)
    {
      \begin{cvitems} % Description(s)
        \item {Description d'un middleware pour les réseaux de capteurs dans le cadre des bâtiments intelligents : mécanismes de reconfiguration dynamique des capteurs par le réseau, gestion et composition de services par ontologie}
      \end{cvitems}
    }

% ---------------------------------------------------------
  \cventry
    {Co-auteur, article de conférence WASC 2015, SAINT JACQUES DE COMPOSTELLE} % Role
    {A New Development Framework for Multi-Core Processor based Smart-Camera Implementations} % Title
    {} % Location
    {Juin 2015} % Date(s)
    {
      \begin{cvitems} % Description(s)
        \item {Article, description d'un framework Linux pour l'éxécution d'algorithmes vidéo sur des caméras intelligentes}
      \end{cvitems}
    }

\end{cventries}
